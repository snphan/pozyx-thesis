Based on the findings in Chapter \ref{chp3}, it was determined that Machine Learning (ML) would be a suitable method for classifying the data obtained from the Pozyx system. Prior to attempting any experimentation, this chapter will conduct a survey of the current ML techniques available for the classification of Time Series Data.


\section{Classic Machine Learning Methods}
\subsection{k-Nearest Neighbours (kNN)}
\subsection{Support Vector Machines (SVM)}
\subsection{Random Forests}

\section{Neural Networks}
\subsection{Deep Nerual Networks}
\subsection{Convultional Neural Networks}
\subsection{Recurrent Neural Networks}
\subsection{Long Short Term Memory}

\section{Large Language Models}


\section{Feature Extraction}
Time series data is a series of data that each consists of a value at an associated timestamp. In terms of the system investigated in this thesis, this could be a series of position in the x-axis measured at time $t_1, t_2, t_3...$ respectively. The data can be said to be indexed by time \cite{yinPredictionAnalysisTime2023} typically in ascending order.

