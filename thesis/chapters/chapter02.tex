% TODO: 
% - Background on ADLs
% - Background on coking
% - Background on trying to assess cooking
% - Market analysis on cooking stuff
% - Background on technologies used to track cooking
% - Background on remote monitoring
%


\section{Overview}
In 2014, over 6 million Canadians (15.6\% of the population) were 65 years old or older. The number of older adults (65+) continues to increase; and by 2030, Statistics Canada expects there to be over 9.5 million adults over the age of 65, comprising 23\% of Canadians \cite{government_of_canada_daily_2014}. Frequently, older adults who wish to age-in-place or in the comfort of their own home must be able to perform their activities of daily living (ADLs) while bearing multiple diseases and syndromes that come with age, such as frailty, impaired cognition, gait and balance problems \cite{tijsen_challenging_2019}. These ADLs may include cooking, bathing, getting into and out of bed, and toileting all of which require complex coordination of the older adult’s cognitive, physical, visual, and perceptual abilities (a complete list may be found in Table \ref{table:ADLs}). Deficits in any of the categories mentioned can impair the older adult’s ability to go about their day. However, it is typically only after an incident or hospitalization that an older adult is assessed for their ability to perform ADLs \cite{wilkinson_comprehensive_2021}. 

The current system presents an opportunity for proactive and preventative medicine through the use of in-home monitoring. Data obtained through monitoring can be used to track functional decline. With this information, older adults, along with their clinician, can plan early interventions and prevent future incidents and hospitalization. 

\renewcommand{\arraystretch}{1.25}
\begin{table}[h!]
    \centering
    \caption{Complete list of Basic ADLs and Instrumental ADLs. Adapted from \cite{edemekong_activities_2022}.}
    \label{table:ADLs}
    \begin{tabular}{ p{0.9in} p{1.5in} p{2.7in} } 
    \hline
    \textbf{Category} & \textbf{Name} & \textbf{Description} \\
    \hline
    \multirow{6}{4em}{Basic ADLs} & Ambulating & Ability to move from one position to another and get around the house. \\ 
    & Feeding & The ability to feed oneself. \\ 
    & Personal Hygiene & The ability to maintain hygiene for oneself. Includes dental, nails and hair hygiene. \\ 
    & Continence & The ability to put on clothes, shoes, pants, etc. \\ 
    & Toileting & The ability to use the toilet appropriately and clean oneself after. \\ 
    \hline
    \multirow{6}{4em}{Instrumental ADLs} & Transportation and Shopping & Ability to arrange transportation for oneself or drive. Also, the ability to procure groceries, clothing and any other daily necessities. \\
    & Managing Finances & Ability to manage assets and pay the bills. \\
    & Meal preparation & End-to-end ability to put a meal onto the table. Involves both purchasing of groceries, cooking, plating and brining to the dining table. \\
    & Housecleaning and home maintenance & Cleaning kitchen after eating and keeping living areas reasonably clean and tidy. The ability to maintain the home and arrange repairs if needed. \\
    & Managing communication with other & The ability to use phone and mail to communicate with others. \\
    & Managing Medication & Ability to obtain medications and take them as directed. \\
    \hline
    \end{tabular}
\end{table}

\clearpage
To better understand the role of in-home monitoring, the diseases and conditions that older adults may face as they age must be understood. Accordingly, the next sections will be organized as follows: first, an overview of the disease or condition including its definition, prevalence and symptoms will be presented; next, current assessment tools for ADLs will be discussed; finally, current attempts at in-home monitoring for ADLs will be investigated as well as what data and how the data is being used.

\section{Background for Diseases and Conditions in Aging}\label{sec:bg_disease}

This section reviews the current literature on the disease and conditions associated that affect the ability to perform ADLs as older adults age. In terms of functionally being able to perform one's ADLs, both physical and cognitive ability are required. Peter et al., in their exploration of ADLs, mention frailty and dementia as conditions limiting ADL performance \cite{edemekong_activities_2022}. In addition to physical and cognitive abilities, sensory decline with hearing, vision and vestibular function (associated with dizziness) occur as one ages \cite{jaul_age-related_2017}. Several chronic conditions may also be present in older adults such as cardiovascular disease including heart failure and ischemic heart disease; diabetes mellitus; osteoarthritis; and osteoporosis \cite{jaul_age-related_2017}.

% NOTE: Diseases/Conditions should have the following structure
% Prevalence
% Definition
% Diagnosis
% How it manifests (signs and symptoms.)

\subsection{Frailty}


Frailty is a syndrome present in 20-50\% of the middle and older aged population (ages 50+ years) \cite{hewitt_prevalence_2018} and is associated with ageing and co-morbidities, but not caused by them \cite{conroy_defining_2009}. Individuals with frailty are at “higher risk for adverse health outcomes such as illnesses, hospitalization, disability and mortality”. To define frailty, there are 2 models: the frailty phenotype and the frailty index \cite{chen_frailty_2014}. The frailty phenotype (also known as the Fried’s Definition of Cardiovascular Health Study) defines frailty as meeting three out of five of the following criteria: “weakness, slowness, low level of physical activity, self-reported exhaustion and unintentional weight loss,” whereas the frailty index uses a comprehensive geriatric assessment to determine cumulative deficits \cite{chen_frailty_2014}. 

\subsection{Dementia}

Worldwide, 47 million people live with dementia, and the number is only expected to increase. By 2050, it is projected that 131 million individuals will be living with dementia \cite{noauthor_diagnosis_nodate}. In the US, the prevalence of dementia in older adults above the age of 68 is 15\% \cite{noauthor_diagnosis_nodate}. Dementia is an umbrella term describing a gradual decline in cognitive abilities in several domains to the point of impairing social or occupational function \cite{noauthor_diagnosis_nodate}. It involves a slow onset and gradual loss of memory, typically paired with the inability to retain new information and ability to perform daily activities \cite{noauthor_diagnosis_nodate}. Prior to diagnosing dementia, there is typically a long history of cognitive decline combined with mention of cognitive decline from close friends and family \cite{noauthor_diagnosis_nodate}. When diagnosing dementia, a common tool that is used is the Mini Mental State Examination (MMSE): a paper test that lists several simple tasks such as spelling a word backwards, and asking about the day of the week, etc. \cite{arevalo-rodriguez_mini-mental_2015}. 

\subsection{Hearing Loss}
The prevalence of hearing loss increases with age and at around 85 years or older, about half of all
older adults experience hearing impairments \cite{desai_trends_2001}. Hearing loss is associated with a decreased quality of life and impairs speech processing. Reduced conversational ability may lead to social isolation which is associated with depression and cognitive decline \cite{jaul_age-related_2017}.

\subsection{Vision Loss}
Similar to hearing loss, the prevalence of vision loss increases with age. In the UK, it was found that 23\% of older adults ages 85-89 had severe vision loss and this increases to 37\% for older adults older than 90 years \cite{evans_prevalence_2002}. Older adults with vision loss are reported to have slower walking speeds and have difficulty doing physical activities \cite{swenor_aging_2020}. Vision loss can also "predict cognitive decline and risk of dementia" \cite{swenor_aging_2020}.

\subsection{Dizziness}
In 2000, Tinetti et al., found that 24\% of their sample American population over age 72 experience dizziness \cite{tinetti_dizziness_2000}. The cause of dizziness in the older population is multifactorial, but the most common cause is peripheral vestibular dysfunction (the part of the ear responsible for balance) \cite{iwasaki_dizziness_2014}. As one ages, the number of vestibular hair cells and neurons decrease which impairs the older adult's ability to sense changes in orientation \cite{iwasaki_dizziness_2014}. As a result, their balance and postural stability deteriorates and they are at higher risk of falling \cite{iwasaki_dizziness_2014}.

\subsection{Heart Failure} 
Heart Failure affects at least 26 million people worldwide and the number of people affected continues to grow \cite{savarese_global_2017}. The American College of Cardiology (ACC) Foundation and American Heart Association (AHA) define Heart failure (HF) as “a complex clinical syndrome that results from any structural or functional impairment of ventricular filling or ejection of blood \cite{ziaeian_epidemiology_2016}.” There are two types of HF: HF with reduced ejection fraction and HF with preserved ejection fraction. HF with reduced ejection fraction occurs at an ejection fraction $\leq$40\% and and HF with preserved ejection fraction occurs at an ejection fraction $\geq$50\% \cite{ziaeian_epidemiology_2016}. Though patients with HF present with a variety symptoms, some include shortness of breath, lethargy, fatigue, reduced exercise tolerance, wheezing, and ankle swelling \cite{watson_clinical_2000}.

\subsection{Ischemic Heart Disease}
Ischemic heart disease (IHD) is also called coronary heart disease (CHD) or coronary artery disease (CAD) \cite{criteria_ischemic_2010} and between the periods of 2003 to 2006 it was estimated that 17.6 million or 7.9\% of Americans age 20 or older had CAD \cite{criteria_ischemic_2010}. Similar to the other conditions, prevalence of IHD increases with age: in 2017 the National Health and Nutrition Examination Survey reported a prevalence of 30.6\% in men and 21.7\% in women over the age of 80 \cite{madhavan_coronary_2018}. IHD manifests due to a condition called atherosclerosis which is a buildup of plaque in the blood vessels \cite{criteria_ischemic_2010}. Atherosclerosis reduces the diameter of the blood vessel which reduces the supply of blood to an organ. The condition when an organ doesn't get enough blood and oxygen is referred to as ischemia. In the case of Ischemic Heart Disease, the heart does not receive enough blood and oxygen \cite{criteria_ischemic_2010}. As a result, IHD may cause symptoms such as fatigue and angina pectoris which is the discomfort that is felt when the heart muscle does not get enough oxygen \cite{criteria_ischemic_2010}.

\subsection{Diabetes Mellitus}
Diabetes occurs when the body cannot control its blood sugar levels \cite{sapra_diabetes_2021}. People who have diabetes are at risk of damaging blood vessels in their eyes, kidneys and nerves \cite{sapra_diabetes_2021}. Diabetes mellitus can come in two main subtypes: type 1 diabetes mellitus (T1DM) or type 2 diabetes mellitus (T2DM) \cite{sapra_diabetes_2021}. T1DM occurs due to the destruction of insulin producing beta-cells from an autoimmune process, while T2DM appears when cells develop a resistance to insulin and fail to use insulin that is being produced. The full extent of T2DM occurs when a cell’s resistance to insulin overtakes the body’s ability to produce insulin \cite{sapra_diabetes_2021}. Diabetes affects 1 in 11 adults. 90\% of adults have T2DM and the remaining 10\% have T1DM. Onset for people with T1DM gradually increases from birth and peaks at ages 4 to 6 years and again from 10 to 14 years \cite{sapra_diabetes_2021}. 
Signs and symptoms of diabetes include being overweight/obese, blurry vision, yeast infections, numbness, and neuropathic pain \cite{sapra_diabetes_2021}. T1DM is diagnosed based on a history of "fasting glucose over 126 mg/dL, random glucose over 200 mg/dL, or hemoglobin A1C exceeding (HbA1C) 6.5\%" \cite{sapra_diabetes_2021}. The early stages of T2DM uses fasting glucose and HbA1C as well, but Sapra and Bhandari do not provide a specific level. A precursor to T2DM, prediabetes can be diagnosed with a glucose level of 100 to 125 mg/dL or a 2-hour post-oral glucose tolerance test (testing blood sugar levels one hour after ingesting 75 g of glucose dissolved into 250 ml to 300 ml of water \cite{noauthor_glucose_2020}) glucose level of 140 to 200 mg/dL \cite{sapra_diabetes_2021}. 

\subsection{Osteoarthritis}
Osteoarthritis affects approximately 27 million Americans \cite{lespasio_hip_2018} and onsets in a third of adults during the typical working age of 45-64 years \cite{hootman_projections_2006}. There are about 315 million visits to the doctor, and 744 000 hospital admissions per year in the US for osteoarthritis. These figures add up to a total of 68 million days off work \cite{prieto-lhambra_osteoarthritis_2014}.
Osteoarthritis is the most common type of joint disorder. It is a disease characterized by the mechanical destruction and failure of a synovial joint \cite{hunter_osteoarthritis_2019}. In order of the most common location, the knee ranks first, followed by the hand and the hip \cite{hunter_osteoarthritis_2019}. High-risk factors or factors that greatly increase the chance of having osteoarthritis include obesity and previous joint injury \cite{prieto-lhambra_osteoarthritis_2014}. 
Osteoarthritis typically presents as debilitating pain, and is accompanied by stiffness, reduced range of motion, joint instability, swelling, muscle weakness, and fatigue \cite{hunter_osteoarthritis_2019}. Clinical diagnosis is based on the symptoms that the patient presents with such as pain, and functional limitations \cite{hunter_osteoarthritis_2019}. Additionally, diagnostic criteria such as those from the American College of Rheumatology may also be used \cite{hunter_osteoarthritis_2019,altman_development_1986}.

\subsection{Osteoporosis}
Osteoporosis is characterized by a reduction in bone mass and structural deterioration inside of the bone \cite{compston_osteoporosis_2019}. It is estimated that over 200 million people have osteoporosis and incidence increases with age \cite{porter_osteoporosis_2023}. Over the age of 80, it is estimated that 70\% of people have osteoporosis \cite{porter_osteoporosis_2023}. This condition leads to an increased risk of bone fractures at all bone locations, with hip and vertebral fractures being historically associated with osteoporosis \cite{compston_osteoporosis_2019}. Typically, there is a long "latent period," where osteoporosis goes undetected, before it manifests clinically as a vertebral, rib or hip fracture \cite{glaser_osteoporosis_1997}. One of the earliest symptoms of osteoporosis occurs as a result vertebral compression fractures \cite{glaser_osteoporosis_1997}. The individual may experience acute back pain at rest or during activity such as bending, standing from a seated position and lifting an object \cite{glaser_osteoporosis_1997}. Individuals with hip fractures experience pain and an inability to bear weight which leads to reduced functional status and quality of life \cite{compston_osteoporosis_2019}. In 2010, 2.7 million hip fractures occurred worldwide \cite{compston_osteoporosis_2019} and as one ages, the incidence of hip fractures increases exponentially \cite{compston_osteoporosis_2019}. 

\section{Traditional Assessment of ADLs}
The traditional assessment of ADLs often involves a pen and paper test. Some tools may require a trained clinician to ask the patient to perform some tasks related to the ADL and observe their performance \cite{pashmdarfard_assessment_2020}. Based on the criteria outlined in the assessment tool, a score is assigned and helps the clinician later identify deficits and plan care accordingly \cite{mcmahon_katz_nodate}. Other tools may require the clinician to check patient medical records, use direct observation or interview the patient directly to assign scores \cite{katz_for_the_association_of_rheumatology_health_professionals_outcomes_measures_task_force_measures_2003}. 

Pashmdarfard and Azad in their systematic review of ADL assessment tools identified 8 tools assessing Basic ADLs and 5 tools assessing Instrumental ADLs \cite{pashmdarfard_assessment_2020}. The Basic ADLs assessment tools are as follows:

\begin{itemize}
    \item Barthel index
    \item Katz Index of Independence in ADLs
    \item Functional Independence Measure
    \item ADL Profile
    \item ADL Questionnaire
    \item Australian Therapy Outcome Measures
    \item Melbourn Low-Vision ADL Index
    \item Self-Assessment Parkinson's disease (PD) Disability Scale 
\end{itemize}

And the Instrumental ADL assessment tools are as follows:

\begin{itemize}
    \item Frenchay Activities Index
    \item ADL Profile Instrumental
    \item Lawton Instrumental ADL Scale
    \item Performance Assessment of Self-Care Skills
    \item Texas Functional Living Scale
\end{itemize}

\clearpage
\subsection{Barthel index}
There are 10 activities related to Basic ADLs that are assessed by the Barthel Index: bowels, bladder, grooming, toilet use, feeding, transfer, mobility, dressing, stairs, and bathing \cite{ahs_barthel_nodate}. Total possible scores range from 0-20 with 20 being highly functional and 0 indicating high disability. Scores can be obtained through self-report, by proxy (reports from someone who is familiar with the individual), or through observation. A version of the Barthel Index used at Alberta Health Services is provided in Figure \ref{fig:barthel-index}.

\begin{figure}[ht]
    \centering
    \includegraphics*[width=0.7\textwidth]{barthel-index}
    \caption{Alberta Health Services (AHS) adaptation of the Barthel Index \cite{ahs_barthel_nodate}}
    \label{fig:barthel-index}
\end{figure}

\clearpage
\subsection{Katz Index of Independence in ADLs}
The Katz Index evaluates 6 categories for Basic ADLs: bathing, dressing, toileting, transfers, continence, and feeding for older adults. Though the Katz Index is binary (either independent or dependent), the evaluator is given 3 choices for each category and must select one for each: independence, intermediate dependence and full dependence \cite{katz_studies_1963}. Figure \ref{fig:katz-index} shows the evaluation form a clinician would use. Adequacy or overall performance in the 6 functions is given as a grade (A, B, C, D, E, F, and other) based on the type of dependence. The description of these grades may be found in Figure \ref{fig:katz-grades}. The results of this assessment can be used to describe functional level at the onset of an illness such as hip fracture \cite{katz_studies_1963}. It may also be used to compare treatment effectiveness between a control and treatment group, acts as a guide for hospital admittance, as well as guide progress during therapy \cite{katz_studies_1963}.

\begin{figure}[ht!]
    \centering
    \begin{subfigure}[t]{.4\textwidth}
        \centering
        \includegraphics*[width=\linewidth]{katz-index}
        \caption{Evaluation form from the Katz Index of Independence in ADLs \cite{katz_studies_1963}}
        \label{fig:katz-index}
    \end{subfigure}
    ~~~
    \begin{subfigure}[t]{.4\textwidth}
        \centering
        \includegraphics*[width=\linewidth]{katz-grades}
        \caption{Grades for determining adequacy outlined in \cite{katz_studies_1963}}
        \label{fig:katz-grades}
    \end{subfigure}
    \caption{The Katz Index of Independence in ADLs \cite{katz_studies_1963}}
\end{figure}

\clearpage
\subsection{Functional Independence Measure}
The Functional Independence Measure (FIM) is frequently used for patients who have experienced stroke \cite{davidson_functional_2014}. FIM is based off of the Barthel Index and contains 18 items that tests independence in "self-care activities, mobility, locomotion, communication, sphincter control, and congnition" \cite{davidson_functional_2014}. Each item is rated from 1 (total assistance) to 7 (independent) for a total score that ranges from 18 to 126--a higher score indicates higher independence \cite{davidson_functional_2014}. If the level of dependence can not be evaluated, then the lowest score of 1 is given \cite{davidson_functional_2014}. Unfortunately, the FIM is proprietary and cannot be reproduced here \cite{noauthor_fimtm_nodate}. 

\clearpage
\subsection{ADL Profile}
The ADL Profile was created to assess ADL function in head injured people whose deficits lie mostly with "complex home and community activities" \cite{dutil_development_1990}. Accordingly, 10 domains were chosen for assessment: personal hygiene, dressing, feeding, health care, meal preparation and home management, and use of public services, transportation, financial management and time management \cite{dutil_development_1990}. Tasks are devised for each domain and the clinician would observe the patient performing the task. Following the completion of a task the clinician would rate the patient on a 3-point ordinal scale with 0 indicating dependence and 2 indicating independence based on their ability to perform the 10 operations common to all of the tasks \cite{dutil_development_1990}. See Figure \ref{fig:adl-profile} for an example of a score sheet for the ADL Profile.

\begin{figure}[ht]
    \centering
    \includegraphics*[width=0.7\textwidth]{adl-profile}
    \caption{Example of a score sheet for some tasks in the ADL profile \cite{dutil_development_1990}. Columns show the operations for each of the tasks.}
    \label{fig:adl-profile}
\end{figure}

\subsection{ADL Questionnaire}
The ADL Questionnaire was developed for individuals with Alzheimer's Disease. It is applicable to a wide range of dementia syndromes and can be used to track functional decline over time \cite{kennedy_activities_2004}. It assess six areas: "self-care, household care, employment and recreation, shopping and money, travel, and communication" \cite{kennedy_activities_2004}. The primary caregiver assigns a score from 0 (no problem) to 3 (unable to perform) for each areas by comparing current level of ability to the level of ability before the onset of dementia" \cite{kennedy_activities_2004}. Impairment is then calculated by formula \ref{eq:adl-quest} (Note that all scores with "9" are excluded because either the patient has never done the activity or the caregiver doesn't know): 

\begin{equation}
    \text{Functional Impairment} = \frac{\text{Sum of all ratings}}{3 * \text{total number of items rated}} * 100\%
    \label{eq:adl-quest}
\end{equation}

Impairment is the rated by ranges: 0-33\% indicates none to mild impairment, 34-66\% indicates moderate impairment and $>66\%$ indicates sever impairment. An excerpt of the questionnaire is provided in Figure \ref{fig:adl-quest}

\begin{figure}[ht]
    \centering
    \includegraphics*[width=0.5\textwidth]{adl-quest}
    \caption{Excerpt of the questions asked in the ADL questionnaire~\cite{kennedy_activities_2004}}
    \label{fig:adl-quest}
\end{figure}

\clearpage
\subsection{Australian Therapy Outcome Measures}
The Australian Therapy Outcome Measures (AuTOM) is based on the UK Therapy Outcome Measures \cite{perry_therapy_2004} and is used for evaluating therapy outcome measures across speech pathology, occupational therapy and physiotherapy. A core scale was developed with categories of impairment of structure or function, activity limitations, participation restriction and well-being each rated from 0 (most impairment) to 5 (no impairment). ADLs seem to reside in the occupational therapy domain which have 12 domains \cite{perry_therapy_2004}:

\begin{enumerate}
    \item learning and applying knowledge
    \item self-care
    \item functional walking and mobility
    \item domestic life: inside house
    \item upper limb use
    \item domestic life: outside house
    \item carrying out daily life tasks and routines
    \item interpersonal interactions and relationships
    \item transfers
    \item work, employment and education
    \item using transport
    \item community life, recreation, leisure and play
\end{enumerate}

\clearpage
\subsection{Melbourn Low-Vision ADL Index}
The Melbourn Low-Vision ADL Index was created to address the need to assess the ADL performance of individuals with vision impairment \cite{haymes_development_2001}. The index contains both task-based observational rating as well as questionnaire self-reported rating. For both ratings a scale of 0 (unsatisfactory) to 4 (satisfactory) was used. For the task-based observational rating, standardized descriptions at each score were developed to reduce the variability. Performance was evaluated based on speed, accuracy and independence and it was suggested that the evaluator time the task to better assess speed of execution \cite{haymes_development_2001}. The list of tasks is shown in Table \ref{tab:MLVAI}.

\begin{table}[ht]
    \centering
    \caption{Observational and questionnaire items in the Melbourn Low-Vision ADL Index \cite{haymes_development_2001}}
    \label{tab:MLVAI}
    \begin{tabular}{ c c }
        \hline
        \textbf{Observational} & \textbf{Questionnaire} \\
        \hline
        Reading newspaper print & Eating \\
        Reading newspaper headlines & Bathing \\
        Reading a letter with typed print & Dressing \\ 
        Using a telephone book & Grooming \\
        Reading an account & Mobility \\
        Reading a medicine label & Housework \\
        Reading packet labels & Shopping \\
        Recognizing faces & Preparing meals \\
        Using a telephone & Managing medication \\
        Writing a check & \\
        Identifying coins & \\
        Pouring & \\
        Naming colors & \\
        Buttoning a shirt & \\
        Threading a sewing needle & \\
        Telling the time: wrist watch & \\
        Telling the time: wall clock & \\
        Reading a digital display & \\
        \hline
    \end{tabular}
\end{table}

\subsection{Self-Assessment PD Disability Scale} 
The Self-Assessment PD Disability Scale is a self-report scale that bases its assessment on the degree of PD's interference on the individual's daily life \cite{brown_accuracy_1989}---and not on the "amplitude of tremors, or speed of foot tapping" \cite{brown_accuracy_1989}. Accordingly, the individual themself should know PD's degree of interference on their daily life. There are a total of 24 items on the questionnaire with 11 assessing gross mobility and 13 assessing fine coordination. The individual is asked to rate themselves on a 5-point scale with 1 indicating no interference and 5 indicating maximum interference from PD \cite{brown_accuracy_1989}. The total score ranges from 24 to 120 with higher scores indicating higher disability \cite{brown_accuracy_1989}. The full list of items can be found in Table \ref{tab:SAPDDS}.

\begin{table}[ht]
    \centering
    \caption{Items for the Self-Assessment PD Disability Scale \cite{brown_accuracy_1989}}
    \label{tab:SAPDDS}
    \begin{tabular}[pos]{ c c }
        \hline
        \textbf{Gross Mobility} & \textbf{Fine Coordination} \\
        \hline
        Getting out of bed & Brushing teeth \\
        Getting out of chair & Washing \\
        Walking around home & Opening tins \\
        Walking outside, eg. to shops & Pouring milk from bottle \\
        Traveling by public transport & Making cup of tea \\
        Walking up stairs & Holding cup and saucer \\
        Walking downstairs & Washing and drying dishes \\
        Getting into bath & Using knife and fork \\
        Getting out of bath & Inserting electrical plug \\
        Getting undressed & Dialing telephone \\
        Picking up object from floor & Holding and reading newspaper \\
        & Writing letter \\
        & Getting dressed \\
        \hline 
    \end{tabular}
\end{table}


\clearpage
\subsection{Frenchay Activities Index}
The Frenchay Activities Index (FAI) was created with 2 goals: first to "provide accurate information for pre-morbid lifestyle for individuals with stroke" and second "record changes in activities following stroke, at specific intervals" \cite{holbrook_activities_1983}. Through factor analysis, 3 factors were identified: "domestic chores", "leisure/work" and "outdoor activities" \cite{holbrook_activities_1983}. The assessment is questionnaire-based and includes 15 items that represent these 3 factors. Each item is given a score of 1 to 4 with 4 indicating more activity. The questionnaire is show in Figure \ref{fig:frenchay}.

\begin{figure}[ht]
    \centering
    \includegraphics*[width=0.55\textwidth]{frenchay}
    \caption{The questionnaire used in the Frenchay Activities Index.}
    \label{fig:frenchay}
\end{figure}

\clearpage
\subsection{ADL Profile Instrumental}
The ADL Profile Instrumental v2 is an updated version of the ADL Profile designed specifically to assess Instrumental ADLs in patients with Traumatic Brain Injury \cite{bottari_iadl_2010}. The scoring system uses a 5-level ordinal scale to finely rate independence \cite{bottari_iadl_2010}. Rather than a separate task for each category of evaluation, a single scenario was given to the patient to assess a total of 8 categories: 6 related to the goal of preparing a meal including "dressing to go outdoors, going to the grocery store, shopping for food, preparing a hot meal for guests, having a meal with guests, and cleaning up after the meal" and 2 related to complex single tasks including "obtaining information, making a budget" \cite{bottari_iadl_2010}. In total, there are 29 scores for 5 tasks with 4 operations and 3 tasks with 3 operations \cite{bottari_iadl_2010}. The scenario prompt involved the following script followed byh observation of the patient performing the task:

\begin{displayquote}
    \textit{"We would like to know how you manage in your everyday activities, that is, activities that  you  generally  do  inside and outside  of  your  home. More specifically, we would like to know, following your accident, if any changes have occurred in your ability to carry out your everyday tasks. Without knowing it, you invited my assistant and me to have lunch with you. Please get ready to receive us. We will assume any incurred expenses for a maximum of \$20. Can you tell me in your own words what I have just explained to you? Would you agree to do this? Can you now tell me in your own words what you are going to do?"} \cite{bottari_iadl_2010}
\end{displayquote}



\clearpage
\subsection{Lawton Instrumental ADL Scale}
\clearpage
\subsection{Performance Assessment of Self-Care Skills}
The Performance Assessment of Self-Care Skills (PASS) is a performance-based observational tool in which the clinician asks the patient to perform the required task for each of the ADLs assessed and evaluates their performance \cite{pashmdarfard_assessment_2020}. PASS assesses an individual’s ability to perform ADLs by judging 3 parameters: independence, safety and adequacy. There are concrete guidelines and identifiers mentioned for scoring each parameter in Performance Assessment of Self-Care Skills \cite{rogers_performance_2014}. For instance, the safety category has a maximum score of 3. At a score of 3, there are no risks observed; at a score of 2, there are minor risks observed, but no assistance is needed; at a score of 1, there are obvious risks to safety and assistance is required to complete a task; Finally, at a score of 0, there are risks to safety of such severity that the task had to be stopped. PASS is used around the world; has been translated to multiple languages including Spanish, Hebrew and Mandarin; has a test-retest reliability of 89\%-90\%; and an inter-observer agreement of 96\%-97\% \cite{chisholm_evaluating_2014} making it a reliable tool for assessing ADLs. However, as much as the PASS is reliable and comprehensive, it is also time-consuming and may be strenuous for the older adults \cite{pashmdarfard_assessment_2020}. 


\subsection{Texas Functional Living Scale}


\section{In-Home Monitoring of ADLs}

% TODO: Choose like 3-4 diseases (frailty dementia cardiovascular disease ) at the end and mention that because their prevlence is higher or something we focus on them.

\section{Narrowing Down}

There is an abundance of assessments that may be used to pinpoint problems with the patient’s ADLs. One of these assessments is the Performance Assessment of Self-Care Skills (PASS) where patients are asked to perform select activities and are assessed by their healthcare provider on their ability to perform each task. 

% TODO: Put in more of the assessments here

% Say here that cooking can be used to assess both physical and cognitive decline.? think of a way to narrow down to cooking, we can just list a bunch of reasons, "there are a lot of directions that we could go" but we chose to go this way blah blah.
Of the ADLs that PASS assesses, cooking was identified to be critical in terms of sustaining a healthy lifestyle in older adults \cite{bouchard_smart_2020} but fraught with risks \cite{yared_cooking_2015}. Older adults also cook frequently, with 53\% of older adults ages 65-80 reported to cook at home 6-7 days a week \cite{malani_joy_2020}. Despite the health benefits and enjoyment gained from cooking, cooking is a dangerous ADL and older adults with comorbidities may have difficulty cooking safely (older adults with cognitive impairments may forget about a stovetop or oven that is on, improper knife use can result in injury and unsafe kitchen environments may exacerbate injury from falls or increase the risk of falls); the kitchen is in second place for the location of most domestic accidents \cite{yared_cooking_2015}.PASS assesses the independence, safety and adequacy of the cooking ADL by splitting the cooking task into categories of oven use, stovetop use, use of sharp utensils, and cleanup after meal preparation each with a list of its own subtasks \cite{rogers_performance_2014}. The concrete classification framework provided by PASS allows for Smart Home (SH) interventions through monitoring and assessment with internet of things (IoT) devices.

Devices that have been used in literature cover a wide range of sensors. Logan and Healy used a modified form of AdaBoost with simple linear weak learners to distinguish meal preparation and eating through accelerometer, video capture, and audio capture data \cite{logan_sensors_2006}. Sarma et al. used Long short-term memory (LSTM) to determine various ADLs from datasets containing motion sensors, door closure sensors and temperature sensors data \cite{sarma_activity_2019}. Chibaudel et al. detected cooking by noting the physical location of the participant and their refrigerator usage through door sensors placed in the refrigerator and motion sensors in the kitchen \cite{mokhtari_smart_2018}. Yordanova et al. used Decision Trees (DT), Computational Causal Behavior Models (CCBM), and Hidden Markov Models (HMM) to process data from the full-scale SPHERE Smart Home system consisting of temperature sensors, humidity sensors, luminosity and motion sensors, water and electricity usage sensors, cameras, and door contact sensors to classify the preparation of a wide range of recipes as "cooking" \cite{yordanova_analysing_2019}.

Although there have been many studies involving the detection of the cooking ADL, few studies assess the quality of cooking and provide feedback to the user. The closest attempts at assessing quality involve quantifying the number of departures from a given task. Cook and Schmitter-Edgecombe used motion sensors, analog sensors for water and stove usage, open/shut sensors for the status of cabinets, and load sensors for the absent/present status of items to assess the quality of ADLs including meal preparation. If a correct sequence of tasks was done and if the task was done efficiently, then an activity is considered as normal. Any significant departures from this "correct sequence" and normal time required to complete each step was only left with a tag of "anomalous" \cite{cook_assessing_2009}. There is no further information with respect to how much of an anomaly the error was or what to do about it.  Similarly, in Menghi et al.'s study, errors were identified in a series of tasks, but nothing was done to evaluate the severity of the error and no feedback was given to the user \cite{mokhtari_smart_2018}. There is no impact on the user because errors were only identified. The user does not receive feedback about what areas need improvement and how to improve, because there was no evaluation using standardized assessments such as PASS.

The literature cites using AI techniques on data collected from IoT devices, from which two issues arise: a huge amount of data will be necessary to produce reliable classification models \cite{sarker_machine_2021}; and, to facilitate collection of these large datasets, data from various IoT device vendors must be easily extractable. To solve these two issues, this project will be part of a recent joint initiative between several Universities across Canada gave rise to the Program to Accelerate Technologies for Homecare (PATH) which will unify SH devices and their various communication protocols (including Bluetooth, Zigbee, and Wi-fi) into a plug-and-play cloud-based system. As the platform matures, PATH will collect data from both home-like labs and at least 350 homes across Canada by partnering with companies in the SH industry. Collecting from this many sources will lead to a huge database of real-world data that will be used to develop and improve AI algorithms for use in monitoring and detection of abnormalities present in the home or the user \cite{path_path_nodate}.

The potential to rapidly collect data on various cooking scenarios provided from PATH, along with concrete classifiable items assessing the quality of cooking provided by the PASS framework can be used to create a novel and robust cooking quality assessment SH system that doubles as a tool for ensuring the safety of older adults in the kitchen and automation of cooking functional assessments for clinicians.

Objective and Methodology: The research project proposes the development of a cooking-focused SH monitoring system on the PATH platform for older adults to ensure that cooking is done safely and to automate cooking assessments for clinicians. This will involve selecting and testing commercially available devices to testing the entire system on patients at the Independent Living Suite (ILS) in the Glenrose Rehabilitation Hospital (GRH). To create a system that is easy to use for older adults and clinicians, they will be consulted at every step in the process. The research project can be broken down into 5 phases:

1.     Research the usability and acceptability of SH devices among older adults with a focus on how the usability and acceptability may be affected by the design of the devices. A UTAUT2 focus group study involving older adults (65+), care providers and clinicians will be conducted to obtain feedback for usability and acceptability;

2.     Research common data analytic tools and frameworks that are relevant for monitoring cooking through the PASS’s criteria. Keywords from the sub tasks outlined in PASS + “Machine Learning, Detection, Assessment” will be searched on the academic databases Scopus, PubMed, Cinahl, IEEE Explore, ISI Web of Sciences, and ACM Digital Library; and on the general web on StackOverflow, TowardsDataScience and TowardsAI;

3.     Select and validate commercially available SH devices tailored to the preferences from the usability study and clinical relevance from the literature review. This step involves searching for previous validation of the devices on Scopus, PubMed, Cinahl, IEEE Explore, ISI Web of Sciences, and ACM Digital Library along with any necessary validation of the devices in-lab by comparing to gold standards such as ECG for Heart Rate;

4.     Apply data analytics methodologies from the literature review to devices selected to assess cooking safety and function. The bulk of these algorithms will be written in Python and any other additional languages deemed necessary from the literature review will be used;

5.     Conduct a clinical pilot-study at the ILS with real-world participants to determine the sensitivity and specificity of devices when detecting the cooking activity and evaluate if users are cooking safely. The developed system will be installed into one living suite at the ILS and a single user will be monitored for the remaining duration of research project. Key outcomes investigated throughout the pilot-study involve cooking ADL detection F-score, cooking safety evaluation compared to clinicians’ judgement and effectiveness of automated SH interventions by comparing independence, safety and adequacy scores before and after.

 

The outcomes of this research project are four-fold: the development of a SH cooking safety system, an automated cooking assessment tool for clinicians, testing and further development of the PATH system, and contribution of data to the PATH system for other researchers.
