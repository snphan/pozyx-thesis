% TODO: 
% - Background on ADLs
% - Background on coking
% - Background on trying to assess cooking
% - Market analysis on cooking stuff
% - Background on technologies used to track cooking
% - Background on remote monitoring
%


\section{Overview}
By 2030, Statistics Canada expects there to be over 9.5 million adults over the age of 65, 
comprising 23\% of Canadians [1]. Older adults (65 years or older), who want to live by themselves 
or age in the comfort of their homes (aging-in-place) must be able to perform their activities of 
daily living (ADLs) while facing declining physical and cognitive abilities [2]. Furthermore, older 
adults may need to face multiple comorbidities that include one or more of the following: 
multiple sclerosis, stroke, Parkinson's disease, dementia, traumatic brain injury 
and ataxia as they age [3]. One important factor in being able to perform 
these ADLs is the older adult's mobility, which Soubra et al. defines as the "[older adult's] 
ability to change [their] position or location or move from one place to another by walking 
and basic ambulation" [4]. Older adults with mobility limitations have a higher risk for falls, 
reduced access to medical services, and poorer psychological health and functional abilities [5]; 
all of which hampers an older adult’s ability to age-in-place. Therefore, it is critical that the 
mobility of the older adult be assessed and tracked regularly to ensure that they have adequate 
mobility to perform all the necessary ADLs as they age [5]. If the older adult does not have 
adequate mobility, the assessments should have the role of highlighting areas of struggles in 
the older adult [4], [6] to a clinician who can suggest interventions or mobility aids.

Currently, mobility assessments take place in the clinic meaning that the older adult must travel 
to the clinic and be physically present with the clinician who instructs the older adults on what 
to do. Soubra et al. identified 31 pen-and-paper assessments that may be used to evaluate mobility 
in older adults [4]. In 2019, the top 5 most frequently cited mobility assessments were the Timed 
Up and Go (TUG), Short Physical Performance Battery (SPPB), Tinetti Performance Oriented Mobility 
Assessment (POMA), the Berg's Balance Scale (BBS), and the Six-Minute Walk Test (6MWT) [4]. 
These assessments can take up to 15 minutes and consist of simple tasks. 

In the Timed up and Go test (TUG), the older adult starts from a seated position and is asked 
to walk 3 meters, turn 180 degrees, walk back to the chair, and sit down [7]. The process is 
timed and the time elapsed can be used as an indicator of functional capacity [7]. It is also 
used in the clinic as a screening tool that suggests patients with a TUG time 12 seconds or 
greater [8] receive further mobility investigation or recommendations for mobility aids [9].

In the Short Physical Performance Battery (SPPB), there are 3 categories of test: a test of 
standing balance by placing their feet side-by-side, semi-tandem and tandem for 10 seconds; 
walking across an 8-foot track; and finally 5 times sit-to-stand [10]. Each category is assigned 
a score of 0 to 4, which could be used to track the lower extremity function of the older adult as 
they age [9]. 

The original Tinetti Performance Oriented Mobility Assessment (POMA) consists of 2 assessment sections: 
balance and gait. The balance section contains 13 items that assess balance such as rising from a chair, 
turning balance, and standing with perturbation and the gait section contains 9 items that are assessed 
by asking the patient to walk down a hallway and back [11]. In practice, a modified POMA is used with 
9 balance tasks and 7 gait tasks scored out of 2 or 3 and totalled to a maximum of 28 [4]. The total 
score can then be used to predict falls, measure mobility impairment, and study the effects of 
interventions [12].  

The Berg Balance Scale (BBS) consists of 14 item that are scored from 0 to 4 with higher scores 
indicating better balance [9], [13]. Items include sitting tasks, standing tasks, and action tasks 
such as retrieving an object from the floor, stepping on a stool, and turning and reaching forward 
while standing [14]. For patients with stroke, a total BBS score of 0 to 20 indicates balance impairment, 
21 to 40 indicated acceptable balance, and 41 to 56 indicates good balance [9]. Generally, scores from the 
BBS can be used to track changes in balance. However, there is a minimal detectable change (MDC) of 2.8 
to 6.6 points (MDC changes based on the score range) that must be considered when concluding significant 
improvement or decline in balance [9].

Finally, in the Six-Minute Walk Test (6MWT), the patient is asked to walk as far as they can in 6 minutes 
[9], [15]. The distance is measured and can be used to assess the patient’s aerobic capacity/endurance [9]. 
Also, a distance travelled less than 338m can indicate an increased risk of all-cause mortality [9]. 

Overall, these pen-and-paper mobility assessments generally seek to evaluate 3 things: fall-risk, 
need for intervention or mobility aids, and change in gait, balance, and transfer ability [4], [9].

In addition to pen-and-paper assessment, there is literature on the use of wearables containing 
inertial measurement units (IMUs) to quantify gait and balance as an alternative to scores, distances, 
or durations from the pen-and-paper assessments.

Zampieri et al. collected gait parameters including stride length, stride velocity, turning velocity, 
and cadence during a Timed up and Go (TUG) test [6]. Data was collected from an IMU on the chest as well 
as gyroscopes attached to the dorsum of each wrist, and each anterior shank (5 sensors in total) [6]. 
Gait parameters collected in the clinic compared to ones collected at home showed that older adults 
with PD performed worse in the home than in the clinic [6]. Furthermore, there was a significant 
difference in the gait parameters between PD patients and able-bodied controls [6] suggesting that 
the gait parameters proposed may be used as part of a mobility assessment and monitoring in older adults. 

In another study, Noamani et al. objectively assessed standing balance by deriving center-of-pressure 
(COP) balance parameters [16] and body center-of-mass (COM) balance parameters [17] from data obtained 
by an IMU placed on the sternum, sacrum, and tibia of the dominant leg during a BBS assessment [18]. 
Balance parameters include root-mean-square distance from mean COP, mean velocity, sway area, median 
frequency in the anterior-posterior, mediolateral, and their resultant distance direction. These balance 
parameters were first compared between older adults and young adults and showed that the measures of COP 
were significantly different [18] indicating decline in balance can be tracked with IMUs. Then, the 
balance parameters were compared to scores from the BBS at admission and discharge for the older adult group. 
Both BBS and balance parameters suggested an improvement at discharge in the older adult group, but it was 
only the balance parameters that could explain what and where the underlying improvements were made 
(eg. reduced sway acceleration and jerkiness in the mediolateral direction) [18]. 

Using sensors have the advantage over pen-and-paper assessments in providing objective data that answer 
why pen-and-paper mobility assessment scores, distances or times were low or high.

Though in-clinic assessments (pen-and-paper or sensors) can provide information on the mobility of 
older adults, results from the in-clinic assessment may be influenced by the white-coat bias 
(being in the clinic affects performance), the Hawthorne effect (being observed affects performance), 
and day-of fatigue, pain, and stress [19] leading to a misrepresentation of performance in unsupervised 
environments such as the home—from Warmerdam et al’s systematic review, it was found that performance 
was overall lower (slower gait speeds, and longer transfer durations) in unsupervised settings compared 
to supervised, clinic, environments [19]; they are time-consuming (commute to the clinic and 
administration of the assessment); and they provide only snapshots of the older adult's mobility 
meaning that clinically relevant in-home events that may include response to dopaminergic treatment 
in Parkinson’s Disease, falls, and freezing [19], [20] may be overlooked. Thus, to assess an older 
adult’s mobility for the purpose of ageing-in-place, unsupervised data collection and analysis are 
essential [19], [21].

To collect data on mobility in unsupervised settings, usage of smart homes, or wearable sensor suites 
have been implemented at various institutions. The Center of Advanced Studies in Adaptive Systems (CASAS) 
created multiple smart homes instrumented with item sensors, motion sensors, and door sensors [22]. 
Kaye et al. placed a series of passive infrared (PIR) sensors 61 cm apart in a single hallway to 
automatically collect trigger events that could be used to calculate the participant's gait speed [23]. 
Schooten et al. used an accelerometer placed at the L5 level on the trunk to obtain gait quality 
characteristics such as walking speed, stride length, stride frequency, intensity, variability, 
smoothness, symmetry, and complexity. These gait quality characteristics were shown to be moderately to 
highly correlated (r > 0.4) with fall incident during monthly check-in with the participant for six to 
twelve months [24]. Sprint et Al. used 3 inertial measurement units (IMUs) placed on the patient's center 
of mass, and both ankles to assess a "standardized ambulation [created during the study] performance task 
called the ambulatory circuit (AC)" [25]. The ambulatory circuit consists of rising from a seated position 
in a chair, walking to the vehicle, transferring into the vehicle, and walking back to and sitting in the 
chair. 16 parameters including walking speed, cadence, shank range of motion, step length, step regularity, 
stride length and step symmetry were calculated. Additionally, duration of events such as sit-to-stand, 
stand-to-sit, and straight walking were calculated, graphed and presented to the clinicians [25]. 
Newland et al. used a 3D depth imaging system to collect stride time, stride length, gait velocity in 
Multiple Sclerosis patients and correlated them to daily symptoms and found that pain and fatigue 
decreased stride length and gait velocity [26]. Finally, Tiger Place uses a sensor suite that consists 
of depth cameras (Microsoft Kinect [27]), motion detectors, and bed sensors to detect activity as well 
as vitals (during sleep) [28]. The depth camera is used to collect gait parameters such as stride length 
and gait speed [27].

The studies in literature have demonstrated that it is possible to collect and analyze data in an 
unsupervised setting with the purpose of monitoring mobility. However, some of the methods require 
many sensors placed on the body [29], [30] and are impractical for long-term implementation in 
unsupervised settings (the older adult will need to put on and take off many sensors everyday). 
In terms of ambient patient mobility monitoring, sensor suites that include cameras may be undesirable 
due to concerns with privacy [26]. Placing motion sensors in series to estimate walking speed may be 
feasible in a small area of the home, but may be impractical if required to be placed around the entire 
home [31]. Additionally, motion sensor systems may have difficulty distinguishing between each person 
in a multi-resident tracking situation without habitual pattern recognition or machine learning [22]. 
Thus, although data may be collected and analyzed in an unsupervised setting, practicality, privacy, 
acceptance, and adherence to the sensor system must be considered. For body-worn sensors, a survey 
by Noury et al. discovered that individuals prefer to have sensors on the wrist the most, followed by 
trunk (chest), belt, ankle, and armpit [32].  A further consideration is integration of data collection 
and analysis into wearables containing sensors that are already available on the market such as the 
Fitbit, Apple Watch, Garmin Watch or sensors can be integrated into accessories that people already 
wear (belts, necklaces, rings). For example, sensors on necklaces and wrist are seen as locations with 
heightened acceptability [33].

In addition to limitations with scalability, privacy, multi-resident tracking, adherence, and 
acceptance with current systems, Warmerdam et al. state that one critical limitation is that 
data from unsupervised environments is challenging to interpret [12]. Misclassification can 
occur from similar IMU signals that occur in situations such as bending over to pick something 
up and a sit-to-stand event; or tremors and teeth brushing [19], [21]. Furthermore, there is not 
yet a standardized way of validating the unsupervised data [19]. Post-analysis of video capture 
can aid with validation, but it is estimated that it will take 1.5h-2h to process and label 
20 minutes of video data [34], [35], which may be hard to implement in studies at scale. All 
mobility parameters are a product of both intrinsic (physiological) and extrinsic (environmental) 
factors. Thus, the context in which the measures of mobility, such as gait asymmetry, are 
assessed in must be considered [19], [21], [36]–[38]. Considering all the limitations with 
current methods of unsupervised mobility data collection and analysis, indoor localization 
combined with wearables sensors may be a scalable solution for capturing context-rich and 
interpretable mobility data in unsupervised settings without compromising privacy.

There are many radio frequency (RF) technologies that can be used to localize an object 
including GPS, WiFi, Bluetooth, and RFID [39]–[42] that requires a tag to placed on the 
person or thing being tracked, and static anchors that use trilateration to determine position 
or returns the position of the anchor as the tag passes by. However, localization used with 
these methods achieves accuracies on the order of meters [39], [42]. An accuracy on the order 
of meters means that the person could be anywhere in a room (2.74m – 4.87m [43]); the system 
will be able to tell that a person is in a room, but not where. Ultra-wideband (UWB) technology, 
however, can have a positioning accuracy of less than 30 centimeters [39] which allows the system 
to tell what furniture or appliance an individual is interacting with within a room. Furthermore, 
each user receives a tag for tracking so multi-resident tracking will be a feature that is 
included with the technology. The context provided by UWB is expected to enhance validation of 
data from wearable IMU. In relation to the situations provided in [19], [21], if it is known the 
older adult is at a static piece of furniture, it is not likely that they are bending over to pick 
something up. Similarly, if it is known that the older adult is at the sink in their washroom, then 
there is a higher chance they are brushing their teeth instead of experiencing tremors.

In line with the goal of continuously assessing mobility at home, this project seeks to fuse 
UWB indoor positioning data with data obtained by a consumer smart watch to assess the 
feasibility of validating unsupervised data and using the system for remote mobility monitoring. 
As part of the feasibility assessment, system-specific algorithms to extract gait and balance 
parameters found previously in literature will also be developed. To address practicality, 
acceptability and adherence, the smartwatch will be worn on the non-dominant wrist, and the 
tag for the UWB system will be worn as a necklace. Current smart watches can measure heart 
rate, steps, and wrist acceleration; and the UWB system can measure position and acceleration 
at the chest.
