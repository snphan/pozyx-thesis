% TODO: 
% - Background on ADLs
% - Background on coking
% - Background on trying to assess cooking
% - Market analysis on cooking stuff
% - Background on technologies used to track cooking
% - Background on remote monitoring
%


\section{Overview}
In 2014, over 6 million Canadians (15.6\% of the population) were 65 years old or older. The number of older adults (65+) continues to increase; and by 2030, Statistics Canada expects there to be over 9.5 million adults over the age of 65, comprising 23\% of Canadians \cite{government_of_canada_daily_2014}. Frequently, older adults who wish to age-in-place or in the comfort of their own home must be able to perform their activities of daily living (ADLs) while bearing multiple diseases and syndromes that come with age, such as frailty, impaired cognition, gait and balance problems \cite{tijsen_challenging_2019}. These ADLs may include cooking, bathing, getting into and out of bed, and toileting all of which require complex coordination of the older adult’s cognitive, physical, visual, and perceptual abilities. Deficits in any of the categories mentioned can impair the older adult’s ability to go about their day. Typically it is only after an incident or hospitalization that an older adult is assessed for their ability to perform ADLs \cite{wilkinson_comprehensive_2021}. 

The current system presents an opportunity for proactive and preventative medicine through the use of in-home monitoring. Data obtained through monitoring can be used to track functional decline.With ths information, older adults, along with their clinician, can plan early interventions and prevent future incidents and hospitalization. The rest of the chapter will be organized as follows: first, an overview of the common diseases and conditions from aging that affect ADLs will be explored; next, current assessment tools for these conditions will discussed; finally, current attempts at in-home monitoring for ADLs will be investigated as well as what data and how the data is being used.

\section{Background for Diseases and Conditions in Aging}

This section reviews the current literature on the disease and conditions associated that affect the ability to perform ADLs as older adults age. In terms of functionally being able to perform one's ADLs, both physical and cognitive ability are required. Peter et al., in their exploration of ADLs, mention frailty and dementia as conditions limiting ADL performance \cite{edemekong_activities_2022}. In addition to physical and cognitive abilities, sensory decline with hearing, vision and vestibular function (associated with dizziness) occur \cite{jaul_age-related_2017}. Several chronic conditions may also be present in older adults such as cardiovascular disease including heart failure, ischemic heart disease, and atherosclerosis; diabetes mellitus; osteoarthritis; and osteoporosis \cite{jaul_age-related_2017}.

\subsection{Frailty}
Frailty is a syndrome present in 20-50\% of the middle and older aged population (ages 50+ years) \cite{hewitt_prevalence_2018} and is associated with ageing and co-morbidities, but not caused by them \cite{conroy_defining_2009}. Individuals with frailty are at “higher risk for adverse health outcomes such as illnesses, hospitalization, disability and mortality” . To define frailty, there are 2 models: the frailty phenotype and the frailty index \cite{chen_frailty_2014}. The frailty phenotype (also known as the Fried’s Definition of Cardiovascular Health Study) defines frailty as meeting three out of five of the following criteria: “weakness, slowness, low level of physical activity, self-reported exhaustion and unintentional weight loss,” whereas the frailty index uses a comprehensive geriatric assessment to determine cumulative deficits \cite{chen_frailty_2014}. There have been several indices for assessing frailty that have been developed including the frailty index (FI), the Survey of Health, Ageing and Retirement in Europe frailty index (SHARE-FI), and the Groningen Frailty indicator. Applications of Smart Homes include encouraging exercise, and proper nutrition. There is also the potential for monitoring of the symptoms of frailty and tracking the progression of the syndrome as outlined in Lin et al.’s study \cite{lin_development_2016}.

\subsection{Dementia}
Dementia is an umbrella term describing a gradual decline in cognitive abilities in several domains to the point of impairing social or occupational function \cite{noauthor_diagnosis_nodate}. Dementia involves a slow onset and gradual loss of memory, typically paired with the inability to retain new information \cite{noauthor_diagnosis_nodate}. Worldwide, 47 million people live with dementia, and the number is only expected to increase. By 2050, it is projected that 131 million individuals will be living with dementia \cite{noauthor_diagnosis_nodate}. When diagnosing dementia, a common assessment tool that is used is the Mini Mental State Examination (MMSE): a paper test that lists several simple tasks such as spelling a word backwards, and asking about the day of the week, etc. \cite{arevalo-rodriguez_mini-mental_2015}. Physical exercise, cognitive stimulation, and a healthy diet are some ways to positively affect cognitive function. Smart Homes may have a role in encouraging exercise, performing cognitively challenging tasks such as crossword puzzles and a healthy diet prior to dementia. For patients already with dementia, Smart Homes may have a role in helping the patient remember things such as reminding the patient to take medication, and helping the patient find misplaced items.

\subsection{Hearing Loss}

\subsection{Vision Loss}

\subsection{Vestibular Function}

\subsection{Cardiovascular Disease}
\subsubsection{Heart Failure} % TODO: Rewrite this section I think Jayden didthis
The American College of Cardiology (ACC) Foundation and American Heart Association (AHA) define Heart failure (HF) as “a complex clinical syndrome that results from any structural or functional impairment of ventricular filling or ejection of blood \cite{ziaeian_epidemiology_2016}.” HF is a clinical syndrome caused by structural and functional defects in myocardium resulting in impairment of ventricular filling or the ejection of blood \cite{inamdar_heart_2016}. The most common cause for HF is reduced left ventricular myocardial function; however, dysfunction of the pericardium, myocardium, endocardium, heart valves or great vessels alone or in combination are also associated with HF \cite{inamdar_heart_2016}.
Heart failure (HF) is affecting at least 26 million people worldwide and is increasing in prevalence \cite{savarese_global_2017}. Currently 5.7 million people in the US have HF, but the projections are worrisome since it is expected that by 2030 more than 8 million people will have this condition, accounting for a 46\% increase in prevalence \cite{savarese_global_2017}.
The major goals of treatment in heart failure are (1) to improve prognosis and reduce mortality and (2) to alleviate symptoms and reduce morbidity by reversing or slowing the cardiac and peripheral dysfunction \cite{inamdar_heart_2016}. For in-hospital patients, in addition to the above goals, other goals of therapy are (1) to reduce the length of stay and subsequent readmission (2) to prevent organ system damage and (3) to appropriately manage the co-morbidities that may contribute to poor prognosis \cite{inamdar_heart_2016}.
‘In-patient’ management of HF: It is advised to admit the patient in the telemetry bed or in ICU and the treatment is based on the following points \cite{inamdar_heart_2016}.
\begin{itemize}
    \item Monitor oxygen, whether PaO2 < 60\% or SaO2 < 90\% \cite{inamdar_heart_2016}.
    \item Provide noninvasive positive pressure ventilation (NIPPV) in the few cases with respiratory distress for respiratory support to avoid subsequent intubation \cite{inamdar_heart_2016}.
    \item Use pharmacological agents depending on the precipitating factors and symptoms/signs for congestion \cite{inamdar_heart_2016}.
\end{itemize}

\subsubsection{Ischemic Heart Disease}
\subsubsection{Artherosclerosis}

\subsection{Diabetes Mellitus}
Diabetes occurs when the body cannot control its blood sugar levels \cite{sapra_diabetes_2021}. People who have diabetes are at risk of damaging blood vessels in their eyes, kidneys and nerves \cite{sapra_diabetes_2021}. Diabetes mellitus can come in two main subtypes: type 1 diabetes mellitus (T1DM) or type 2 diabetes mellitus (T2DM) \cite{sapra_diabetes_2021}. T1DM occurs due to the destruction of insulin producing beta-cells from an autoimmune process, while T2DM appears when cells develop a resistance to insulin and fail to use insulin that is being produced. The full extent of T2DM occurs when a cell’s resistance to insulin overtakes the body’s ability to produce insulin \cite{sapra_diabetes_2021}. Diabetes affects 1 in 11 adults. 90\% of adults have T2DM and the remaining 10\% have T1DM. Onset for people with T1DM gradually increases from birth and peaks at ages 4 to 6 years and again from 10 to 14 years \cite{sapra_diabetes_2021}. 
Signs of diabetes include the following:
\begin{itemize}
    \item overweight/obese \cite{sapra_diabetes_2021}
    \item acanthosis nigricans \cite{sapra_diabetes_2021}
    \item blurry vision \cite{sapra_diabetes_2021}
    \item yeast infections \cite{sapra_diabetes_2021}
    \item numbness \cite{sapra_diabetes_2021}
    \item neuropathic pain \cite{sapra_diabetes_2021}
\end{itemize}
Management of type 2 diabetes depends on the hemoglobin A1c levels of the person. The higher the A1c, level the greater the intervention. From an A1c level of 5.7\%-6.5\% management involves encouraging lifestyle changes, weight loss and increased exercise. For levels of 6.5\% to 9.0\% metformin is added in, and as the A1c levels increase, a second antihyperglycemic drug is added. Finally, once the A1c increases past 9.0\%, a basal insulin or prandial insulin is added to the prescription \cite{reusch_management_2017}.
Smart homes (SH) can be used to monitor blood glucose and make intelligent decisions, such as alerting a physician and providing the physician with blood glucose data when the SH detects blood glucose levels that are not normal \cite{rghioui_smart_2020}. Another use of SH in diabetes is to remind the patient to take their insulin and medication \cite{norell_pejner_smart_2019}.


\subsection{Osteoarthritis}
Osteoarthritis is the most common type of joint disorder. It is a disease characterized by the mechanical destruction and failure of a synovial joint \cite{hunter_osteoarthritis_2019}. High-risk factors or factors that greatly increase the chance of having osteoarthritis include obesity and previous joint injury \cite{prieto-lhambra_osteoarthritis_2014}. Osteoarthritis affects approximately 27 million Americans \cite{lespasio_hip_2018} and onsets in a third of adults during the typical working age of 45-64 years \cite{hootman_projections_2006}. There are about 315 million visits to the doctor, and 744 000 hospital admissions per year in the US for osteoarthritis. These figures add up to a total of 68 million days off work \cite{prieto-lhambra_osteoarthritis_2014}.
Osteoarthritis is accompanied by pain and stiffness, which is a driver in the clinical management of osteoarthritis \cite{hunter_osteoarthritis_2019}. Strategies recommended by the American College of Rheumatology begin with slowing the progression of the disease. For instance, if an individual is overweight, they can decrease the progression of osteoarthritis by decreasing their weight. If an individual’s joint is misaligned, then braces can be used to improve alignment and reduce joint loads \cite{prieto-lhambra_osteoarthritis_2014}. If an individual has muscle weakness, then strength training may be beneficial for slowing down osteoarthritis. 
In the case that these methods do not appreciably slow the progression of osteoarthritis or improve the symptoms of osteoarthritis, pharmacologic intervention can begin with simple over-the-counter pain relievers such as Tylenol \cite{prieto-lhambra_osteoarthritis_2014}. Vitamins and supplements can also be considered; although the side effects are under investigation, chondroitin sulphate and strontium ranelate have promising results in improving the quality of bone and slowing the progression of osteoarthritis \cite{prieto-lhambra_osteoarthritis_2014}. Nonsteroidal anti-inflammatory drugs (NSAIDS) or opioids can be administered to alleviate symptoms. However, in cases where symptoms are too severe, the last line of defense for managing osteoarthritis is surgery and may involve a total joint replacement.
Since obesity is a high-risk factor for causing osteoarthritis \cite{hunter_osteoarthritis_2019}, Smart Homes can have a role in promoting lifestyle changes such as reminding the patient to exercise or eat healthy. Also, wearables that can measure gait may be used to track the progression of the disease and provide data to clinicians to assist with management of osteoarthritis \cite{choi_mhealth_2019}.


\subsection{Osteoporosis}



% TODO: Choose like 3-4 diseases (frailty dementia cardiovascular disease ) at the end and mention that because their prevlence is higher or something we focus on them.

\section{Assessing Conditions and Diseases}

There is an abundance of assessments that may be used to pinpoint problems with the patient’s ADLs. One of these assessments is the Performance Assessment of Self-Care Skills (PASS) where patients are asked to perform select activities and are assessed by their healthcare provider on their ability to perform each task. 

% TODO: Put in more of the assessments here

PASS assesses an individual’s ability to perform ADLs by judging 3 parameters: independence, safety and adequacy. There are concrete guidelines and identifiers mentioned for scoring each parameter in Performance Assessment of Self-Care Skills \cite{rogers_performance_2014}. For instance, the safety category has a maximum score of 3. At a score of 3, there are no risks observed; at a score of 2, there are minor risks observed, but no assistance is needed; at a score of 1, there are obvious risks to safety and assistance is required to complete a task; Finally, at a score of 0, there are risks to safety of such severity that the task had to be stopped. PASS is used around the world; has been translated to multiple languages including Spanish, Hebrew and Mandarin; has a test-retest reliability of 89\%-90\%; and an inter-observer agreement of 96\%-97\% \cite{chisholm_evaluating_2014} making it a reliable tool for assessing ADLs. 


Of the ADLs that PASS assesses, cooking was identified to be critical in terms of sustaining a healthy lifestyle in older adults \cite{bouchard_smart_2020} but fraught with risks \cite{yared_cooking_2015}. Older adults also cook frequently, with 53\% of older adults ages 65-80 reported to cook at home 6-7 days a week \cite{malani_joy_2020}. Despite the health benefits and enjoyment gained from cooking, cooking is a dangerous ADL and older adults with comorbidities may have difficulty cooking safely (older adults with cognitive impairments may forget about a stovetop or oven that is on, improper knife use can result in injury and unsafe kitchen environments may exacerbate injury from falls or increase the risk of falls); the kitchen is in second place for the location of most domestic accidents \cite{yared_cooking_2015}.PASS assesses the independence, safety and adequacy of the cooking ADL by splitting the cooking task into categories of oven use, stovetop use, use of sharp utensils, and cleanup after meal preparation each with a list of its own subtasks \cite{rogers_performance_2014}. The concrete classification framework provided by PASS allows for Smart Home (SH) interventions through monitoring and assessment with internet of things (IoT) devices.

Devices that have been used in literature cover a wide range of sensors. Logan and Healy used a modified form of AdaBoost with simple linear weak learners to distinguish meal preparation and eating through accelerometer, video capture, and audio capture data \cite{logan_sensors_2006}. Sarma et al. used Long short-term memory (LSTM) to determine various ADLs from datasets containing motion sensors, door closure sensors and temperature sensors data \cite{sarma_activity_2019}. Chibaudel et al. detected cooking by noting the physical location of the participant and their refrigerator usage through door sensors placed in the refrigerator and motion sensors in the kitchen \cite{mokhtari_smart_2018}. Yordanova et al. used Decision Trees (DT), Computational Causal Behavior Models (CCBM), and Hidden Markov Models (HMM) to process data from the full-scale SPHERE Smart Home system consisting of temperature sensors, humidity sensors, luminosity and motion sensors, water and electricity usage sensors, cameras, and door contact sensors to classify the preparation of a wide range of recipes as "cooking" \cite{yordanova_analysing_2019}.

Although there have been many studies involving the detection of the cooking ADL, few studies assess the quality of cooking and provide feedback to the user. The closest attempts at assessing quality involve quantifying the number of departures from a given task. Cook and Schmitter-Edgecombe used motion sensors, analog sensors for water and stove usage, open/shut sensors for the status of cabinets, and load sensors for the absent/present status of items to assess the quality of ADLs including meal preparation. If a correct sequence of tasks was done and if the task was done efficiently, then an activity is considered as normal. Any significant departures from this "correct sequence" and normal time required to complete each step was only left with a tag of "anomalous" \cite{cook_assessing_2009}. There is no further information with respect to how much of an anomaly the error was or what to do about it.  Similarly, in Menghi et al.'s study, errors were identified in a series of tasks, but nothing was done to evaluate the severity of the error and no feedback was given to the user \cite{mokhtari_smart_2018}. There is no impact on the user because errors were only identified. The user does not receive feedback about what areas need improvement and how to improve, because there was no evaluation using standardized assessments such as PASS.

The literature cites using AI techniques on data collected from IoT devices, from which two issues arise: a huge amount of data will be necessary to produce reliable classification models \cite{sarker_machine_2021}; and, to facilitate collection of these large datasets, data from various IoT device vendors must be easily extractable. To solve these two issues, this project will be part of a recent joint initiative between several Universities across Canada gave rise to the Program to Accelerate Technologies for Homecare (PATH) which will unify SH devices and their various communication protocols (including Bluetooth, Zigbee, and Wi-fi) into a plug-and-play cloud-based system. As the platform matures, PATH will collect data from both home-like labs and at least 350 homes across Canada by partnering with companies in the SH industry. Collecting from this many sources will lead to a huge database of real-world data that will be used to develop and improve AI algorithms for use in monitoring and detection of abnormalities present in the home or the user \cite{path_path_nodate}.

The potential to rapidly collect data on various cooking scenarios provided from PATH, along with concrete classifiable items assessing the quality of cooking provided by the PASS framework can be used to create a novel and robust cooking quality assessment SH system that doubles as a tool for ensuring the safety of older adults in the kitchen and automation of cooking functional assessments for clinicians.

Objective and Methodology: The research project proposes the development of a cooking-focused SH monitoring system on the PATH platform for older adults to ensure that cooking is done safely and to automate cooking assessments for clinicians. This will involve selecting and testing commercially available devices to testing the entire system on patients at the Independent Living Suite (ILS) in the Glenrose Rehabilitation Hospital (GRH). To create a system that is easy to use for older adults and clinicians, they will be consulted at every step in the process. The research project can be broken down into 5 phases:

1.     Research the usability and acceptability of SH devices among older adults with a focus on how the usability and acceptability may be affected by the design of the devices. A UTAUT2 focus group study involving older adults (65+), care providers and clinicians will be conducted to obtain feedback for usability and acceptability;

2.     Research common data analytic tools and frameworks that are relevant for monitoring cooking through the PASS’s criteria. Keywords from the sub tasks outlined in PASS + “Machine Learning, Detection, Assessment” will be searched on the academic databases Scopus, PubMed, Cinahl, IEEE Explore, ISI Web of Sciences, and ACM Digital Library; and on the general web on StackOverflow, TowardsDataScience and TowardsAI;

3.     Select and validate commercially available SH devices tailored to the preferences from the usability study and clinical relevance from the literature review. This step involves searching for previous validation of the devices on Scopus, PubMed, Cinahl, IEEE Explore, ISI Web of Sciences, and ACM Digital Library along with any necessary validation of the devices in-lab by comparing to gold standards such as ECG for Heart Rate;

4.     Apply data analytics methodologies from the literature review to devices selected to assess cooking safety and function. The bulk of these algorithms will be written in Python and any other additional languages deemed necessary from the literature review will be used;

5.     Conduct a clinical pilot-study at the ILS with real-world participants to determine the sensitivity and specificity of devices when detecting the cooking activity and evaluate if users are cooking safely. The developed system will be installed into one living suite at the ILS and a single user will be monitored for the remaining duration of research project. Key outcomes investigated throughout the pilot-study involve cooking ADL detection F-score, cooking safety evaluation compared to clinicians’ judgement and effectiveness of automated SH interventions by comparing independence, safety and adequacy scores before and after.

 

The outcomes of this research project are four-fold: the development of a SH cooking safety system, an automated cooking assessment tool for clinicians, testing and further development of the PATH system, and contribution of data to the PATH system for other researchers.
