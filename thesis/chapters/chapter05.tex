Continuing from the findings in Chapter \ref{chp3}, the 9H anchor configuration
was used to perform a preliminary classification of activities in the 
kitchen. Steps performed in making a sandwich were broken down and organized
into Setup, Preparation, Cooking, and Finishing steps, Figure \ref{fig:sandwichbreakdown}. 

\begin{figure}[ht]
    \centering
    \includegraphics*[width=\textwidth]{makingsandwich}
    \caption{Task decomposition of making a sandwich.}
    \label{fig:sandwichbreakdown}
\end{figure}

From the actions shown in Figure \ref{fig:sandwichbreakdown}, actions 
with distinct location or patterns were selected as classes for classification.
OPENFRIDGE, OPENFREEZER, and GETPLATE were selected as classes from the Setup
category, washing hands/vegetables/fruits/using the kitchen sink were grouped 
into a WASHHANDS category, and SLICETOMATO were selected as a class. Finally,
All intermediary transitions or motionless segments were grouped into a 
UNDEFINED category. Single trials were performed to collect data for each of 
these classes. 

\clearpage
\section{Experimental Protocol}
\subsection{Setup}
Several points were enforced to ensure that the training dataset captures
the variation in action sufficiently when classifying data from right-handed individuals.

\begin{itemize}
    \item Pozyx Tag is mounted on the right wrist (Figure \ref{fig:wristsensor}).
    \item Initial position for each of the single trials are not marked. Participant
    will be able to choose a location from which they can perform the action comfortably
    without moving their feet.
    \item An action starts when the individual contacts the appliance or furniture.
    For SLICETOMATO the action starts when an individual starts slicing the 
    tomato and ends when they stop slicing the tomato. Motions such as picking up the
    knife and getting in position to slice were considered transitions and labelled 
    as UNDEFINED. 
\end{itemize}


\begin{figure}[ht]
    \centering
    \includegraphics*[width=0.4\textwidth]{wristsensor}
    \caption{Pozyx tag mounted on the wrist. The participant is performing the 
    OPENFRIDGE task
    }
    \label{fig:wristsensor}
\end{figure}

\subsection{Data Collection}
Custom Python stopwatch scripts were created to accurately label periods of 
transitions (quiet standing + getting into position for the action) and the action.
An example of the data collected is shown in Figure \ref{fig:openfridgedata}. 
For each action there is a quiet standing period at the beginning and end. 
OPENFRIDGE, OPENFREEZER, OPENPLATE, WASHHANDS each had 5 repetitions for each trial.
SLICETOMATO contained 3 slices to conserve the amount of tomato.
Each action had a total of 5 trials. 

\begin{figure}[ht]
    \centering
    \includegraphics*[width=\textwidth]{openfridgedata}
    \caption{Labelled position data of the OPENFRIDGE action. Note that the "quiet standing"
    periods do not consist entirely of quiet standing, but also include traces of 
    transitions from getting into the correct position to perform the action. 
    }
    \label{fig:openfridgedata}
\end{figure}

Since the Pozyx Tag contained a BNO055 chip, in addition to 3D Position data, the
tags were able to capture inertial data including Accelerometer Data, 
Linear Accelerometer Data, Angular Velocity Data, and the orientation.

Data for each of the actions that relate to making a sandwich were collected from 
2 participants.

\section{Feature Extraction}
An initial sliding window with a width of 2 seconds and a stride length of 1 second
was used to ensure that enough feature vectors could be extracted from the SLICETOMATO dataset.
An example of a windows taken for the OPENFRIDGE action and UNDEFINED action 
are shown in Figure \ref{fig:windows}.

\begin{figure}[ht]
    \centering
    \begin{subfigure}{\textwidth}
        \includegraphics*[width=\textwidth]{windowing1}
        \caption{}
    \end{subfigure}
    \begin{subfigure}{\textwidth}
        \includegraphics*[width=\textwidth]{windowing3}
        \caption{}
    \end{subfigure}
    \caption{Obtaining windows from the OPENFRIDGE dataset. The green vertical
    lines section off a 2-second window. (a) A window
    labelled OPENFRIDGE. (b) A window labelled UNDEFINED.
    }
    \label{fig:windows}
\end{figure}

From each window, basic statistical measures over the entire window were taken. These
measures include the MEAN, MEDIAN, MODE (to 5cm for position), MAX, MIN, and STD
of the entire window. 
From each window of data, there were a total of 3 (axes) * 5 (types of data) * 6 (statistical measures) = 90 Features 

From the entire timeseries dataset, 2773 feature vectors were extracted.
Refer to Table \ref{tab:countsfeatvect} for the breakdown of counts for each label.

\begin{table}[!htbp]
  \centering
  \caption{Count of the occurrences of each action.}
  \label{tab:countsfeatvect}
  \begin{tabular}{ll}
    \toprule
    \thead{Action} & \thead{Count} \\
    \midrule
    UNDEFINED   & 1518 \\
    SLICETOMATO & 197 \\
    WAHSHANDS   & 316 \\
    OPENFRIDGE  & 239 \\
    OPENFREEZER & 203 \\
    GETPLATE    & 300 \\
    \bottomrule
  \end{tabular}
\end{table}

\clearpage
\section{Model Selection}
A 60:40 split was used to train and test the model selected. Several models 
were chosen including Linear Support Vector Machine, Radial Support Vector Machine,
K-Nearest Neighbors, Decision Trees and Random Forests. As this was a pilot study in determining the 
feasibility of classification of the fine-grained actions involved
in making a sandwich, rigorous parameter tuning and feature selection 
were neglected and the defaults from the sklearn Python package were used.

\section{Results}
The confusion matrices from each model are output in Figures \ref{fig:cm-svm-lin}-\ref{fig:cm-rf}.
Total accuracy was reported as well as the sensitivity, specificity, and precision of each class 
were reported. These measures are calculated as follows:

\begin{equation}
    \text{Accuracy} = \frac{\text{All} \: TP}{N}
\end{equation}

\begin{equation}
    \text{Sensitivity} = \frac{TP}{TP + FN}
\end{equation}

\begin{equation}
    \text{Precision} = \frac{TP}{TP + FP}
\end{equation}

\begin{equation}
    \text{Specificity} = \frac{TN}{TN + FP}
\end{equation}
Where $N$ is the number of samples $TP$ are True Positives, 
$TN$ are true negatives, $FP$ are false positives and $FN$ are false negatives.

\begin{figure}[ht]
    \centering
    \includegraphics*[height=0.8\textheight]{cm-svm-lin}
    \caption{Test confusion matrix using the Support Vector Classifier with a Linear Kernel}
    \label{fig:cm-svm-lin}
\end{figure}


\begin{figure}[ht]
    \centering
    \includegraphics*[height=0.8\textheight]{cm-svm-rbf}
    \caption{Test confusion matrix using the Support Vector Classifier with a Radial Kernel}
    \label{fig:cm-svm-rbf}
\end{figure}

\begin{figure}[ht]
    \centering
    \includegraphics*[height=0.8\textheight]{cm-svm-poly}
    \caption{Test confusion matrix using the Support Vector Classifier with a Polynomial Kernel}
    \label{fig:cm-svm-poly}
\end{figure}

\begin{figure}[ht]
    \centering
    \includegraphics*[height=0.8\textheight]{cm-5nn}
    \caption{Test confusion matrix using the K-Nearest Neighbors Classifier}
    \label{fig:cm-5nn}
\end{figure}

\begin{figure}[ht]
    \centering
    \includegraphics*[height=0.8\textheight]{cm-decision}
    \caption{Test confusion matrix using the Decision Tree Classifier}
    \label{fig:cm-decision}
\end{figure}

\begin{figure}[ht]
    \centering
    \includegraphics*[height=0.8\textheight]{cm-rf}
    \caption{Test confusion matrix using the Random Forests Classifier}
    \label{fig:cm-rf}
\end{figure}

\clearpage
\section{Discussion}
Table \ref{tab:accuracies} summarizes the accuracies obtained from each model.

\begin{table}[!htbp]
  \centering
  \caption{Accuracy of each Model}
  \label{tab:accuracies}
  \begin{tabular}{ll}
    \toprule
    \thead{Model Name} & \thead{Accuracy (\%)} \\
    \midrule
    SVM Linear      & 97.7 \\
    SVM Radial      & 88.1 \\
    SVM Polynomial  & 93.1 \\
    kNN             & 97.4 \\
    Decision Tree   & 97.5 \\
    Random Forests  & 99.3 \\
    \bottomrule
  \end{tabular}
\end{table}

With the exception of the Radial SVM, all of the models perform well
achieving an accuracy of somewhere in the high 90s. In day-to-day
activities, there is a disproportionately higher number of the 
UNDEFINED class compared to the other "action" classes signifying the 
presence of class imbalance. If a classifier 
guesses all UNDEFINED it can obtain an accuracy of 599/1110 = 54\%. Thus,
accuracies taken around 54\% should be interpreted with caution. Other 
metrics such as the Sensitivity, Precision and Specificity have been provided
to address this class imbalance. Sensitivity is the rate at which the classifier
predicts a $TP$, Precision is the fraction of predictions that are actually
true, and Specificity is the rate at which the classifier predicts a $TN$.
Of all the models, the Random Forests Classifier at the default settings
seem to the best in terms of Accuracy and Precision, Sensitivity, and Specificity
for all classes. 

The performance of these models in the real-time will need to be tested 
and quantified before any conclusions can be made. A high accuracy 
is promising, but may also be indicative of overfitting which means 
that the model will not be able to generalize variation experienced in the 
real world. In later sections, more fine-grained actions will be considered,
models will be more rigorously tuned, and the performance in real-time
will be investigated.

