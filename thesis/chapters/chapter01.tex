This thesis project will investigate how to detect fine-grained action within the meal preparation activity of daily living 
(ADL) in the home without the use of privacy-intruding cameras. ADLs are common activities that an individual 
performs inside their homes. These include walking around, eating, dressing, personal hygiene, toileting, 
transportation, meal preparation, house cleaning, and managing medication. The meal preparation ADL was chosen as the 
main focus because cooking is a uniquely enjoyable activity while being procedurally dense. Meal preparation can include 
the following actions: opening the fridge, retrieving ingredients, cutting vegetables, and assembling the ingredients. 
Monitoring these actions may be used as part of a health monitoring program by enabling the assessment of the presence and 
length of each individual step in a goal-orientated activity. This information may help guide interventions and track 
the effectiveness of interventions in clinical populations such as people with dementia or frailty. 

Several studies have used features from inertial data to classify these fine-grained action…

It is hypothesized that inclusion of context to this inertial data, such as hyper accurate 
indoor localization (down to 30 cm), can result in more accurate and more reliable 
classification of fine-grained ADLs. 
The system used for indoor localization is the Pozyx Creator Kit which provides a wrist mounted 
wearable that can obtain data at a maximum of 60 Hz \cite{noauthor_creator_nodate}. This data includes position relative to a 
floorplan, and inertial data from a BNO055 which outputs 3D Acceleration, 3D angular velocity, 
3D Linear Acceleration, and the Heading, Pitch and Roll.

Prior to any experiments related to classification of these cooking actions, the optimal 
configuration of the system that provides reliable position data had to be investigated. 
The next section details the different attempts at changing the configuration of the Pozyx 
Creator Kit to obtain the most reliable positioning data in X, Y, and Z at a satisfactory 
sampling rate. 
